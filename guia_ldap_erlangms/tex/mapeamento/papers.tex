%% ATENÇÃO: Gerado automaticamente com o programa bibliografia/papers.py
%% Para atualizar: python3.4 papers.py

\makeatletter
    \newcommand{\thickhline}{%
        \noalign {\ifnum 0=`}\fi \hrule height 1pt
        \futurelet \reserved@a \@xhline
    }
    \newcolumntype{"}{@{\vrule width 1pt}}
    \makeatother


\begin{table*}[htbp]
\scalefont{0.72}
\caption {Lista das publicações selecionadas para análise.} \label{table:artigos}
\centering % centering table

\newcolumntype{L}[1]{>{\raggedright\let\newline\\\arraybackslash\hspace{0pt}}p{#1}}
\newcolumntype{C}[1]{>{\centering\let\newline\\\arraybackslash\hspace{0pt}}m{#1}}
\newcolumntype{R}[1]{>{\raggedleft\let\newline\\\arraybackslash\hspace{0pt}}m{#1}}

\definecolor{Gray}{gray}{0.85}
\definecolor{LightCyan}{rgb}{0.88,1,1}

\setlength{\extrarowheight}{0.5pt}


\begin{tabular}{"| L{4.98cm} | L{4.98cm} | L{4.98cm} |"}


\thickhline%\toprule

[1] Bennett et al. Software
maintenance and evolution: a roadmap. ICSE 2000.
&
[2] Erlikh et al. Leveraging legacy system
dollars for e-business. IT professional 2000.
&
[3] Bisbal et al. Legacy information systems:
Issues and directions. IEEE Software 1999.

\\ \hline

[4] Bennett et al. Legacy systems: coping with success. IEEE 1995.
&
[5] Sneed et al. Integrating legacy software into a service oriented architecture. CSMR 2006.
&
[6] Lewis et al. Service-oriented migration and reuse technique (smart). IEEE 2005.
\\ \hline

[7] Canfora et al. A wrapping approach for migrating legacy system interactive functionalities to service oriented architectures. JSS 2008.
&
[8] Zhang et al. Incubating services in legacy systems for architectural migration. IEEE 2004.
&
[9] Canfora et al. Migrating interactive legacy systems to web services. CSMR 2006.
\\ \hline

[10] Bianchi et al. Iterative reengineering of legacy systems. IEEE Transactions 2003.
&
[11] Canfora et al. Decomposing legacy programs: A first step towards migrating to client–server platforms. JSS 2000.
&
[12] Weiderman et al. Approaches to Legacy System Evolution. SEI 1997.
\\ \hline

[13] Wu et al. The butterfly methodology: A gateway-free approach for migrating legacy information systems. IEEE 1997.
&
[14] Sneed et al. Encapsulation of legacy software: A technique for reusing legacy software components. ASE 2000.
&
[15] Comella-Dorda et al. A survey of legacy system modernization approaches. SEI 2000.
\\ \hline

[16] Ransom et al. A method for assessing legacy systems for evolution. SMR 1998.
&
[17] Fleurey et al. Model-driven engineering for software migration in a large industrial context. Springer 2007.
&
[18] Comella-Dorda et al. A survey of black-box modernization approaches for information systems. ICSM 2000.
\\ \hline

[19] Lewis et al. SMART: Analyzing the reuse potential of migrating legacy components to SOA. SEI 2008.
&
[20] Serrano et al. Reengineering legacy systems for distributed environments. JSS 2002.
&
[21] Lucia et al. Developing legacy system migration methods and tools for technology transfer. SPE 2008.
\\ \hline

[22] Lewis et al. Service-oriented architecture and its implications for software maintenance and evolution. FoSM 2008.
&
[23] Moore et al. Migrating legacy user interfaces to the internet: shifting dialogue initiative. WCRE 2000.
&
[24] Warren et al. The renaissance of legacy systems: method support for software-system evolution. Springer 2012.
\\ \hline

[25] Visaggio et al. Value-based decision model for renewal processes in software maintenance. ASE 2000.
&
[26] Weiderman et al. Implications of distributed object technology for reengineering. SEI 1997.
&
[27] Cetin et al. Legacy migration to service-oriented computing with mashups. ICSEA 2007. 
\\ \hline

[28] Colosimo et al. Evaluating legacy system migration technologies through empirical studies. Science Direct 2009.
&
[29] Erradi et al. Evaluation of strategies for integrating legacy applications as services in a service oriented architecture. SCC 2006.
&
[30] Chung et al. Service-oriented software reengineering: SoSR. IEEE 2007.
\\ \hline

[31] Litoiu et al. Migrating to web services: a performance engineering approach. SMR 2004.
&
[32] Liu et al. Reengineering legacy systems with RESTful web service. COMPSAC 2008.
&
[33] O'Brien et al. Supporting migration to services using software architecture reconstruction. IEEE 2005.
\\ \hline

[34] Chiang et al. Wrapping legacy systems for use in heterogeneous computing environments. Science Direct 2001.
&
[35] Smith et al. Migration of legacy assets to service-oriented architecture environments. IEEE 2007.
&
[36] Wu et al. Legacy systems migration-a method and its tool-kit framework. IEEE 1997.
\\ \hline

[37] Bovenzi et al. Enabling legacy system accessibility by web heterogeneous clients. CSMR 2003.
&
[38] Li et al. Migrating legacy information systems to web services architecture. JDM 2007.
&
[39] Zdun et al. Reengineering to the web: A reference architecture. CSMR 2002.
\\ \hline

[40] Cetin et al. A mashup-based strategy for migration to service-oriented computing. IEEE 2007.
&
[41] Zhang et al. A black-box strategy to migrate GUI-based legacy systems to web services. IEEE 2008.
&
[42] Serrano et al. Evolutionary migration of legacy systems to an object-based distributed environment. ICSM 1999.
\\ \hline

[43] Qiao et al. Bridging legacy systems to model driven architecture. COMPSAC 2003.
&
[44] Canfora et al. Software evolution in the era of software services. IEEE 2004.&
\\ \hline
\thickhline%\toprule


\end{tabular}
\end{table*}

