\subsection{An\'{a}lise Relacionada à Terceira Questão de Pesquisa}

\begin{enumerate}[(QP3)]

\item Quais as razões que levam as organizações a modernizarem os seus sistemas legados?

\end{enumerate}

De acordo com as publicações analisadas, observa-se que, de maneira geral, existem 5 razões para um projeto de modernização dos sistemas legados, conforme descrito a seguir: 

\begin{itemize}

\item \textbf{Falta de integração entre os sistemas --} É cada vez maior a demanda por sistemas integrados nas 
organizações para prover a automação dos processos de negócio e permitir uma gestão com maior racionamento de recursos. 
Contudo, de acordo com~\cite{S3_Bisbal:1999, S29_Chung:2007, S31_LiuYan:2008}, muitos sistemas legados não foram projetados 
para serem integrados, razão pela qual as organizações investem em projetos de modernização, visando a integração dos fluxos e 
compartilhamento das funcionalidades de negócio existentes. Conforme 
observado em~\cite{S4_bennett1995legacy, S15_Comella-DordaASurvey2000, S2_erlikh:2000, Umar:2009}, 
existem vários outros benefícios obtidos com a integração dos sistemas tais como, a diminuição da quantidade de implementa\c c\~{o}es de regras 
de neg\'{o}cio duplicadas, a reutilização de soluções de software já desenvolvidas e a 
diminuição dos custos com o desenvolvimento.

\item \textbf{Tornarem-se mais flexíveis a mudanças --} Uma das principais razões que levam as organizações a buscarem a modernização 
dos seus sistemas legados é para torná-los mais flexíveis 
a mudanças~\cite{S4_bennett1995legacy, S01_bennett2000software, S3_Bisbal:1999, S15_Comella-DordaASurvey2000, S13_ransom1998method, 
S12_WeidermanApproaches:1997}. 
Para Bennet~\cite{S4_bennett1995legacy}, o \textit{time-to-market} dos sistemas é prioridade número 1 para a maioria das organizações. 
Em~\cite{S3_Bisbal:1999, S4_bennett1995legacy}, é relatado que muitos sistemas começam a ser chamados de sistemas legados, justamente porque 
passam a resistir mais as modificações que devem ser feitas, o que dificulta as evoluções no software que precisam ser implementadas para os negócios das organizações. 
Este pensamento vai de acordo com o que diz~\cite{S2_erlikh:2000}: ``A maioria das empresas querem transformar suas aplicações para atender a novos negócios e demandas, 
mas porque os sistemas legados tendem a ser de difícil controle, monolíticos e inflexíveis, muitas empresas consideram a modernização como em algum lugar entre improvável e impossível''. 

\item \textbf{Minimizar o custo de manutenção com o legado --} Reduzir o custo de manutenção dos sistemas legados é um dos grandes desafios para as organizações. 
Para~\cite{S4_bennett1995legacy, S15_Comella-DordaASurvey2000, S13_ransom1998method, S12_WeidermanApproaches:1997}, os 
sistemas legados são sistemas usualmente críticos para os negócios mas que o custo dispendido para mantê-los funcionando é quase sempre injustificável. 
\cite{S10_canfora2000decomposing} enfatiza que a manutenção frequentemente monopoliza os esforços das organizações, pois tais atividades, 
incluindo correção de erros, adaptações e melhorias em geral, consomem entre 50\% e 70\% do orçamento de um software típico. 
Adicionalmente,~\cite{S3_Bisbal:1999, S13_wu1997butterfly:1997} destacam que a falta de documentação e do conhecimento interno dos sistemas é um 
dos motivos dos altos custos bem como da demora nas manutenções para correção de falhas nos softwares. Portanto, como afirma~\cite{S4_bennett1995legacy}, há um dilema: 
de um lado, o sistema é muito valioso e uma substituição pode ser muito cara para ser contemplada. E, por outro lado, o sistema torna-se muito caro para manter e a
s demandas das organizações podem não ser sustentadas. Além do mais,~\cite{S23_warren2002renaissance} enfatiza que 
a substituição de um sistema legado poderia incorrer no risco da perda de informações críticas do negócio da organização.

\item \textbf{Falta de conhecimento e domínio do legado --} Como mencionado anteriormente, a falta de conhecimento e domínio nos sistemas legados 
é uma das justificativas para um projeto de modernização. De acordo com~\cite{S4_bennett1995legacy, S3_Bisbal:1999}, o entendimento 
do funcionamento de um sistema é visto como um dos requisitos para fazer as modificações que são requeridas pelas organizações. Tem sido reportado na 
literatura que uma parte substancial do tempo envolvido na compreensão de um sistema legado é na localização dos conceitos de domínio do problema a ser 
resolvido no código fonte para então serem feitas as implementações~\cite{S4_bennett1995legacy, clements2002documenting}. Assim, a compreensão dos sistemas legados representa 
um dos problemas de pesquisa centrais da literatura, conforme salienta~\cite{S4_bennett1995legacy}. 
Muitas pesquisas são despendidas para identificar
formas de obter um melhor entendimento do sistema que é vital para qualquer 
exercício de evolução como sugere~\cite{S01_bennett2000software, S13_ransom1998method}. Além 
disso,~\cite{S4_bennett1995legacy} reitera que existem os problemas de gerenciamento de pessoal. A maioria dos engenheiros de software preferem trabalhar em 
novos sistemas em vez de manter sistemas antigos e de tecnologia ultrapassada. E \textit{skills} necessários podem estar cada vez mais reduzidos e escassos, decorrente da saída dos profissionais que conhecem esses sistemas das organizações.

\item \textbf{Propenso a falhas --} De acordo com~\cite{S4_bennett1995legacy}, sem documentação atualizada, as evoluções nos softwares são efetuadas usando 
o próprio código fonte como documentação confiável. Aliado aos problemas de gerenciamento de pessoal, acredita-se que os sistemas podem ser alvo de 
falhas pela falta de conhecimento e domínio nesses sistemas. Apesar disso,~\cite{S4_bennett1995legacy} afirma que os sistemas podem ser muito confiáveis e 
responsivos para as necessidades dos usuários, bem mais do que um novo sistema que viesse a substituir o atual. No entanto, existe um sentimento percebido por várias organizações que os sistemas legados podem falhar devido a falta de especialistas e/ou suporte (empresas que mantêm e dão treinamento em tecnologias legadas). Nesse sentido, uma falha poderia causar um sério impacto para as organizações, como afirma~\cite{S3_Bisbal:1999}.

\end{itemize}
