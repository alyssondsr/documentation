\section{Introdução}

A modernização dos sistemas 
legados tem lugar quando as 
tradicionais práticas de manutenção deixam de 
atender \`{a}s organizações. Entre os objetivos que se buscam com a modernização, 
podem-se citar a redução dos custos com manutenção, maior integração entre os sistemas e torná-los mais flexíveis às mudanças, 
de forma a prolongar sua vida útil~\cite{S4_bennett1995legacy,S3_Bisbal:1999,S15_Comella-DordaASurvey2000}. 

Do ponto de vista das organizações, os sistemas legados correspondem às aplicações que sustentam o funcionamento 
negocial de uma instituição e que consolidam a maior parte das informações corporativas~\cite{S3_Bisbal:1999}. 
Nesse contexto, a modernização dos sistemas legados ganha cada vez mais importância 
para a Universidade de Brasília (UnB), uma vez que, 
nos últimos 20 anos, uma gama considerável de sistemas foi desenvolvida. 
São sistemas com um arcabouço de regras de negócio que foram sendo 
construídas ao longo dos anos e que são de vital importância para o pleno funcionamento da Universidade. 
Entretanto, durante o ciclo de vida desses sistemas, 
ocorreram várias revisões para mantê-los alinhados com as necessidades da Instituição, 
tornando-os mais rígidos e inflexíveis, 
a ponto de serem de difícil manutenção. 

Os sistemas da Universidade dividem-se em três áreas de negócio: área de gestão acadêmica, administrativa e de pessoal. 
A maioria desses sistemas estão sob diferentes linguagens de programação, arquiteturas e plataformas; e não conversam entre
si, a não ser, por meio do banco de dados. Durante muitos anos, a linguagem de programação predominante foi o Visual Basic. 
Dois dos sistemas mais importantes estão escritos nessa linguagem: Sistema de
Informações e Gestão Acadêmica (SIGRA) e o Sistema de
Informações de Pessoal (SIPES). Os demais sistemas foram
desenvolvidos em VB.NET, C\#, PHP, ASP e Java (plataforma atual). 

Este artigo apresenta os resultados iniciais de uma disserta\c c\~{a}o 
de mestrado que tem como objetivo propor uma abordagem para apoiar 
a modernização dos sistemas legados da Universidade, 
com vistas a diminuir a duplicidade de implementa\c c\~{a}o 
de regras de negócio entre as aplicações (um dos problemas mais 
críticos relacionados à qualidade do software desenvolvido pelo 
CPD). Essa abordagem deve atender a alguns requisitos, tais como: (a) 
seguir uma estrat\'{e}gia orientada a serviço e  
aderente a arquitetura \textit{Representational State Transfer} (REST); 
(b) apresentar uma curva de aprendizagem aceitável, para que possa ser 
realmente implantada no CPD; e oferecer bons mecanismos de disponibilidade, 
escalabilidade e monitoramento.

Mais especificamente, aqui são descritos os principais resultados alcançados 
até esse momento:

\begin{itemize}
\item a caracterização do termo modernização de software, alcançado com a 
condução de um mapeamento sistemático (Seção~\ref{ms}). 

\item a implementação de um protótipo arquitetural de um barramento 
orientado a serviços (Seção~\ref{arquitetura}), cujas decisões de projeto atende aos 
requisitos de alta disponibilidade, escalabilidade e monitoramento.  
\end{itemize} 

Este artigo também apresenta o planejamento de um 
estudo empírico (Seção~\ref{estudo-empirico}) que está sendo 
conduzido com o envolvimento dos alunos matriculados em uma disciplina da pós-graduação 
relacionada a análise estática e engenharia reversa de software. 



